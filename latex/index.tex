Structure

\href{https://travis-ci.com/open-space-collective/open-space-toolkit-physics}{\tt } \href{https://codecov.io/gh/open-space-collective/open-space-toolkit-physics}{\tt } \href{https://open-space-collective.github.io/open-space-toolkit-physics}{\tt } \href{https://badge.fury.io/gh/open-space-collective%2Fopen-space-toolkit-physics}{\tt } \href{https://badge.fury.io/py/open-space-toolkit-physics}{\tt } \href{https://opensource.org/licenses/Apache-2.0}{\tt }

Physical units, time, reference frames, environment modeling.



{\itshape Gravitational field anomaly between E\+G\+M96 and W\+G\+S84 models.}

\subsection*{Getting Started}

Want to get started? This is the simplest and quickest way\+:

\href{https://mybinder.org/v2/gh/open-space-collective/open-space-toolkit/master?urlpath=lab/tree/notebooks}{\tt }

{\itshape Nothing to download or install! This will automatically start a \href{https://jupyterlab.readthedocs.io/en/stable/}{\tt Jupyter\+Lab} environment in your browser with Open Space Toolkit libraries and example notebooks ready to use.}

\subsubsection*{Alternatives}

\paragraph*{Docker Images}

\href{https://www.docker.com/}{\tt Docker} must be installed on your system.

\subparagraph*{i\+Python}

The following command will start an \href{https://ipython.org/}{\tt i\+Python} shell within a container where the O\+S\+Tk components are already installed\+:


\begin{DoxyCode}
docker run -it openspacecollective/open-space-toolkit-physics-python
\end{DoxyCode}


Once the shell is up and running, playing with it is easy\+:


\begin{DoxyCode}
\textcolor{keyword}{from} \hyperlink{namespaceostk_1_1physics}{ostk.physics} \textcolor{keyword}{import} Environment \textcolor{comment}{# Environment modeling class}
\textcolor{keyword}{from} \hyperlink{namespaceostk_1_1physics_1_1time}{ostk.physics.time} \textcolor{keyword}{import} Instant \textcolor{comment}{# Instant class}
\textcolor{keyword}{from} ostk.physics.coordinate \textcolor{keyword}{import} Frame \textcolor{comment}{# Reference frame class}

environment = Environment.default() \textcolor{comment}{# Bootstrap a default environment}

moon = environment.access\_object\_with\_name(\textcolor{stringliteral}{'Moon'}) \textcolor{comment}{# Access Moon}

environment.set\_instant(Instant.now()) \textcolor{comment}{# Set environment to present time}

moon.get\_position\_in(Frame.ITRF()) \textcolor{comment}{# Position of the Moon in ITRF}
moon.get\_axes\_in(Frame.ITRF()) \textcolor{comment}{# Axes of the Moon in ITRF}
\end{DoxyCode}


By default, O\+S\+Tk fetches the ephemeris from J\+PL, Earth Orientation Parameters (E\+OP) and leap second count from I\+E\+RS.

As a result, when running O\+S\+Tk for the first time, it may take a minute to fetch all the necessary data.

{\itshape Tip\+: Use tab for auto-\/completion!}

\subparagraph*{Jupyter\+Lab}

The following command will start a \href{https://jupyterlab.readthedocs.io/en/stable/}{\tt Jupyter\+Lab} server within a container where the O\+S\+Tk components are already installed\+:


\begin{DoxyCode}
docker run --publish=8888:8888 openspacecollective/open-space-toolkit-physics-jupyter
\end{DoxyCode}


Once the container is running, access \href{http://localhost:8888/lab}{\tt http\+://localhost\+:8888/lab} and create a Python 3 Notebook.

\subsection*{Installation}

\subsubsection*{C++}

The binary packages are hosted using \href{https://github.com/open-space-collective/open-space-toolkit-physics/releases}{\tt Git\+Hub Releases}\+:


\begin{DoxyItemize}
\item Runtime libraries\+: {\ttfamily open-\/space-\/toolkit-\/physics-\/\+X.\+Y.\+Z-\/1.\+x86\+\_\+64-\/runtime}
\item C++ headers\+: {\ttfamily open-\/space-\/toolkit-\/physics-\/\+X.\+Y.\+Z-\/1.\+x86\+\_\+64-\/devel}
\item Python bindings\+: {\ttfamily open-\/space-\/toolkit-\/physics-\/\+X.\+Y.\+Z-\/1.\+x86\+\_\+64-\/python}
\end{DoxyItemize}

\paragraph*{Debian / Ubuntu}

After downloading the relevant {\ttfamily .deb} binary packages, install\+:


\begin{DoxyCode}
apt install open-space-toolkit-physics-*.deb
\end{DoxyCode}


\paragraph*{Fedora / Cent\+OS}

After downloading the relevant {\ttfamily .rpm} binary packages, install\+:


\begin{DoxyCode}
dnf install open-space-toolkit-physics-*.rpm
\end{DoxyCode}


\subsubsection*{Python}

Install from \href{https://pypi.org/project/open-space-toolkit-physics/}{\tt Py\+PI}\+:


\begin{DoxyCode}
pip install open-space-toolkit-physics
\end{DoxyCode}


\subsection*{Documentation}

Documentation is available here\+:


\begin{DoxyItemize}
\item \href{https://open-space-collective.github.io/open-space-toolkit-physics}{\tt C++}
\item \href{./bindings/python/docs}{\tt Python}
\end{DoxyItemize}

$<$details$>$

The library exhibits the following structure\+:


\begin{DoxyCode}
├── Units
│   ├── Length
│   ├── Mass
│   ├── Time
│   ├── Temperature
│   ├── Electric Current
│   ├── Luminous Intensity
│   └── Derived
│       ├── Angle
│       ├── Solid Angle
│       ├── Frequency
│       ├── Force
│       ├── Pressure
│       ├── Area
│       ├── Volume
│       └── Information
├── Time
│   ├── \hyperlink{namespaceostk_1_1physics_1_1time_adf23d37bd8641fb76a0e98ab46a70df7}{Scale} (UTC, TT, TAI, UT1, TCG, TCB, TDB, GMST, GPST, GST, GLST, BDT, QZSST, IRNSST)
│   ├── Instant
│   ├── Duration
│   ├── Interval
│   ├── Date
│   ├── Time
│   └── DateTime
├── Coordinate
│   ├── Transform
│   └── Frame (ECI, ECEF, NED, LVLHGD, LVLHGDGT, ...)
├── Geographic
│   ├── Position
│   ├── Area
│   ├── Volume
│   ├── Coordinate Reference System (CRS)
│   └── Universal Transverse Mercator (UTM)
└── Environment
    ├── Constants
    ├── Object
    │   └── Celestial
    ├── Ephemerides
    │   ├── Analytical
    │   ├── Tabulated
    │   ├── SOFA
    │   └── SPICE (JPL)
    ├── Gravity
    │   ├── Barycentric
    │   ├── Earth Gravitational Model 1996 (EGM96)
    │   └── Earth Gravitational Model 2008 (EGM2008)
    ├── Atmospheric
    │   ├── Exponential
    │   ├── USSA1976
    │   ├── Jacchia Roberts
    │   └── NRLMSISE00
    ├── Magnetic
    │   ├── Dipole
    │   ├── World Magnetic Model 2010 (WMM2010)
    │   ├── World Magnetic Model 2015 (WMM2015)
    │   ├── Enhanced Magnetic Model 2010 (EMM2010)
    │   ├── Enhanced Magnetic Model 2015 (EMM2015)
    │   ├── International Geomagnetic Reference Field 11 (IGRF11)
    │   └── International Geomagnetic Reference Field 12 (IGRF12)
    ├── Radiation
    │   └── Sun Static
    └── Stars
        └── Hipparcos
\end{DoxyCode}


$<$/details$>$

\subsection*{Tutorials}

Tutorials are available here\+:


\begin{DoxyItemize}
\item \href{./tutorials/cpp}{\tt C++}
\item \href{./tutorials/python}{\tt Python}
\end{DoxyItemize}

\subsection*{Settings}

The following environment variables can be set\+:

\tabulinesep=1mm
\begin{longtabu} spread 0pt [c]{*{2}{|X[-1]}|}
\hline
\rowcolor{\tableheadbgcolor}\textbf{ Environment Variable }&\textbf{ Default Value  }\\\cline{1-2}
\endfirsthead
\hline
\endfoot
\hline
\rowcolor{\tableheadbgcolor}\textbf{ Environment Variable }&\textbf{ Default Value  }\\\cline{1-2}
\endhead
{\ttfamily O\+S\+T\+K\+\_\+\+P\+H\+Y\+S\+I\+C\+S\+\_\+\+C\+O\+O\+R\+D\+I\+N\+A\+T\+E\+\_\+\+F\+R\+A\+M\+E\+\_\+\+P\+R\+O\+V\+I\+D\+E\+R\+S\+\_\+\+I\+E\+R\+S\+\_\+\+M\+A\+N\+A\+G\+E\+R\+\_\+\+M\+O\+DE} &{\ttfamily Manual} \\\cline{1-2}
{\ttfamily O\+S\+T\+K\+\_\+\+P\+H\+Y\+S\+I\+C\+S\+\_\+\+C\+O\+O\+R\+D\+I\+N\+A\+T\+E\+\_\+\+F\+R\+A\+M\+E\+\_\+\+P\+R\+O\+V\+I\+D\+E\+R\+S\+\_\+\+I\+E\+R\+S\+\_\+\+M\+A\+N\+A\+G\+E\+R\+\_\+\+L\+O\+C\+A\+L\+\_\+\+R\+E\+P\+O\+S\+I\+T\+O\+RY} &{\ttfamily ./.open-\/space-\/toolkit/physics/coordinate/frame/providers/iers} \\\cline{1-2}
{\ttfamily O\+S\+T\+K\+\_\+\+P\+H\+Y\+S\+I\+C\+S\+\_\+\+C\+O\+O\+R\+D\+I\+N\+A\+T\+E\+\_\+\+F\+R\+A\+M\+E\+\_\+\+P\+R\+O\+V\+I\+D\+E\+R\+S\+\_\+\+I\+E\+R\+S\+\_\+\+M\+A\+N\+A\+G\+E\+R\+\_\+\+L\+O\+C\+A\+L\+\_\+\+R\+E\+P\+O\+S\+I\+T\+O\+R\+Y\+\_\+\+L\+O\+C\+K\+\_\+\+T\+I\+M\+E\+O\+UT} &{\ttfamily 60} \\\cline{1-2}
{\ttfamily O\+S\+T\+K\+\_\+\+P\+H\+Y\+S\+I\+C\+S\+\_\+\+C\+O\+O\+R\+D\+I\+N\+A\+T\+E\+\_\+\+F\+R\+A\+M\+E\+\_\+\+P\+R\+O\+V\+I\+D\+E\+R\+S\+\_\+\+I\+E\+R\+S\+\_\+\+M\+A\+N\+A\+G\+E\+R\+\_\+\+R\+E\+M\+O\+T\+E\+\_\+\+U\+RL} &{\ttfamily \href{ftp://cddis.gsfc.nasa.gov/pub/products/iers/}{\tt ftp\+://cddis.\+gsfc.\+nasa.\+gov/pub/products/iers/}} \\\cline{1-2}
{\ttfamily O\+S\+T\+K\+\_\+\+P\+H\+Y\+S\+I\+C\+S\+\_\+\+E\+N\+V\+I\+R\+O\+N\+M\+E\+N\+T\+\_\+\+E\+P\+H\+E\+M\+E\+R\+I\+D\+E\+S\+\_\+\+S\+P\+I\+C\+E\+\_\+\+E\+N\+G\+I\+N\+E\+\_\+\+M\+O\+DE} &{\ttfamily Manual} \\\cline{1-2}
{\ttfamily O\+S\+T\+K\+\_\+\+P\+H\+Y\+S\+I\+C\+S\+\_\+\+E\+N\+V\+I\+R\+O\+N\+M\+E\+N\+T\+\_\+\+E\+P\+H\+E\+M\+E\+R\+I\+D\+E\+S\+\_\+\+S\+P\+I\+C\+E\+\_\+\+M\+A\+N\+A\+G\+E\+R\+\_\+\+L\+O\+C\+A\+L\+\_\+\+R\+E\+P\+O\+S\+I\+T\+O\+RY} &{\ttfamily ./.open-\/space-\/toolkit/physics/environment/ephemerides/spice} \\\cline{1-2}
{\ttfamily O\+S\+T\+K\+\_\+\+P\+H\+Y\+S\+I\+C\+S\+\_\+\+E\+N\+V\+I\+R\+O\+N\+M\+E\+N\+T\+\_\+\+E\+P\+H\+E\+M\+E\+R\+I\+D\+E\+S\+\_\+\+S\+P\+I\+C\+E\+\_\+\+M\+A\+N\+A\+G\+E\+R\+\_\+\+R\+E\+M\+O\+T\+E\+\_\+\+U\+RL} &{\ttfamily \href{https://naif.jpl.nasa.gov/pub/naif/generic_kernels/}{\tt https\+://naif.\+jpl.\+nasa.\+gov/pub/naif/generic\+\_\+kernels/}} \\\cline{1-2}
{\ttfamily O\+S\+T\+K\+\_\+\+P\+H\+Y\+S\+I\+C\+S\+\_\+\+E\+N\+V\+I\+R\+O\+N\+M\+E\+N\+T\+\_\+\+G\+R\+A\+V\+I\+T\+A\+T\+I\+O\+N\+A\+L\+\_\+\+E\+A\+R\+T\+H\+\_\+\+M\+A\+N\+A\+G\+E\+R\+\_\+\+E\+N\+A\+B\+L\+ED} &{\ttfamily false} \\\cline{1-2}
{\ttfamily O\+S\+T\+K\+\_\+\+P\+H\+Y\+S\+I\+C\+S\+\_\+\+E\+N\+V\+I\+R\+O\+N\+M\+E\+N\+T\+\_\+\+G\+R\+A\+V\+I\+T\+A\+T\+I\+O\+N\+A\+L\+\_\+\+E\+A\+R\+T\+H\+\_\+\+M\+A\+N\+A\+G\+E\+R\+\_\+\+L\+O\+C\+A\+L\+\_\+\+R\+E\+P\+O\+S\+I\+T\+O\+RY} &{\ttfamily ./.open-\/space-\/toolkit/physics/environment/gravitational/earth} \\\cline{1-2}
{\ttfamily O\+S\+T\+K\+\_\+\+P\+H\+Y\+S\+I\+C\+S\+\_\+\+E\+N\+V\+I\+R\+O\+N\+M\+E\+N\+T\+\_\+\+G\+R\+A\+V\+I\+T\+A\+T\+I\+O\+N\+A\+L\+\_\+\+E\+A\+R\+T\+H\+\_\+\+M\+A\+N\+A\+G\+E\+R\+\_\+\+R\+E\+M\+O\+T\+E\+\_\+\+U\+RL} &{\ttfamily \href{https://sourceforge.net/projects/geographiclib/files/gravity-distrib/}{\tt https\+://sourceforge.\+net/projects/geographiclib/files/gravity-\/distrib/}} \\\cline{1-2}
{\ttfamily O\+S\+T\+K\+\_\+\+P\+H\+Y\+S\+I\+C\+S\+\_\+\+E\+N\+V\+I\+R\+O\+N\+M\+E\+N\+T\+\_\+\+M\+A\+G\+N\+E\+T\+I\+C\+\_\+\+E\+A\+R\+T\+H\+\_\+\+M\+A\+N\+A\+G\+E\+R\+\_\+\+E\+N\+A\+B\+L\+ED} &{\ttfamily false} \\\cline{1-2}
{\ttfamily O\+S\+T\+K\+\_\+\+P\+H\+Y\+S\+I\+C\+S\+\_\+\+E\+N\+V\+I\+R\+O\+N\+M\+E\+N\+T\+\_\+\+M\+A\+G\+N\+E\+T\+I\+C\+\_\+\+E\+A\+R\+T\+H\+\_\+\+M\+A\+N\+A\+G\+E\+R\+\_\+\+L\+O\+C\+A\+L\+\_\+\+R\+E\+P\+O\+S\+I\+T\+O\+RY} &{\ttfamily ./.open-\/space-\/toolkit/physics/environment/magnetic/earth} \\\cline{1-2}
{\ttfamily O\+S\+T\+K\+\_\+\+P\+H\+Y\+S\+I\+C\+S\+\_\+\+E\+N\+V\+I\+R\+O\+N\+M\+E\+N\+T\+\_\+\+M\+A\+G\+N\+E\+T\+I\+C\+\_\+\+E\+A\+R\+T\+H\+\_\+\+M\+A\+N\+A\+G\+E\+R\+\_\+\+R\+E\+M\+O\+T\+E\+\_\+\+U\+RL} &{\ttfamily \href{https://sourceforge.net/projects/geographiclib/files/magnetic-distrib/}{\tt https\+://sourceforge.\+net/projects/geographiclib/files/magnetic-\/distrib/}} \\\cline{1-2}
\end{longtabu}
\subsection*{Setup}

\subsubsection*{Development Environment}

Using \href{https://www.docker.com}{\tt Docker} for development is recommended, to simplify the installation of the necessary build tools and dependencies. Instructions on how to install Docker are available \href{https://docs.docker.com/install/}{\tt here}.

To start the development environment\+:


\begin{DoxyCode}
make start-development
\end{DoxyCode}


This will\+:


\begin{DoxyEnumerate}
\item Build the {\ttfamily openspacecollective/open-\/space-\/toolkit-\/physics-\/development} Docker image.
\item Create a development environment container with local source files and helper scripts mounted.
\item Start a {\ttfamily bash} shell from the {\ttfamily ./build} working directory.
\end{DoxyEnumerate}

If installing Docker is not an option, you can manually install the development tools (G\+CC, C\+Make) and all required dependencies, by following a procedure similar to the one described in the \href{./docker/development/Dockerfile}{\tt Development Dockerfile}.

\subsubsection*{Build}

From the {\ttfamily ./build} directory\+:


\begin{DoxyCode}
cmake ..
make
\end{DoxyCode}


{\itshape Tip\+: {\ttfamily helpers/build.\+sh} simplifies building from within the development environment.}

\subsubsection*{Test}

To start a container to build and run the tests\+:


\begin{DoxyCode}
make test
\end{DoxyCode}


Or to run them manually\+:


\begin{DoxyCode}
./bin/open-space-toolkit-physics.test
\end{DoxyCode}


{\itshape Tip\+: {\ttfamily helpers/test.\+sh} simplifies running tests from within the development environment.}

\subsection*{Dependencies}

\tabulinesep=1mm
\begin{longtabu} spread 0pt [c]{*{4}{|X[-1]}|}
\hline
\rowcolor{\tableheadbgcolor}\textbf{ Name }&\textbf{ Version }&\textbf{ License }&\textbf{ Link  }\\\cline{1-4}
\endfirsthead
\hline
\endfoot
\hline
\rowcolor{\tableheadbgcolor}\textbf{ Name }&\textbf{ Version }&\textbf{ License }&\textbf{ Link  }\\\cline{1-4}
\endhead
Pybind11 &2.\+6.\+1 &B\+S\+D-\/3-\/\+Clause &\href{https://github.com/pybind/pybind11}{\tt github.\+com/pybind/pybind11} \\\cline{1-4}
Eigen &3.\+3.\+7 &M\+P\+L2 &\href{http://eigen.tuxfamily.org/index.php}{\tt eigen.\+tuxfamily.\+org} \\\cline{1-4}
I\+AU S\+O\+FA &2018-\/01-\/30 &\href{http://www.iausofa.org/tandc.html}{\tt S\+O\+FA Software License} &\href{http://www.iausofa.org}{\tt www.\+iausofa.\+org} \\\cline{1-4}
S\+P\+I\+CE Toolkit &N0066 &\href{https://naif.jpl.nasa.gov/naif/rules.html}{\tt N\+A\+IF} &\href{https://naif.jpl.nasa.gov/naif/toolkit.html}{\tt naif.\+jpl.\+nasa.\+gov/naif/toolkit.html} \\\cline{1-4}
Geographic\+Lib &1.\+49 &M\+IT &\href{https://geographiclib.sourceforge.io}{\tt geographiclib.\+sourceforge.\+io} \\\cline{1-4}
Core &master &Apache License 2.\+0 &\href{https://github.com/open-space-collective/open-space-toolkit-core}{\tt github.\+com/open-\/space-\/collective/open-\/space-\/toolkit-\/core} \\\cline{1-4}
I/O &master &Apache License 2.\+0 &\href{https://github.com/open-space-collective/open-space-toolkit-io}{\tt github.\+com/open-\/space-\/collective/open-\/space-\/toolkit-\/io} \\\cline{1-4}
Mathematics &master &Apache License 2.\+0 &\href{https://github.com/open-space-collective/open-space-toolkit-mathematics}{\tt github.\+com/open-\/space-\/collective/open-\/space-\/toolkit-\/mathematics} \\\cline{1-4}
\end{longtabu}
\subsection*{Contribution}

Contributions are more than welcome!

Please read our \hyperlink{_c_o_n_t_r_i_b_u_t_i_n_g_8md}{contributing guide} to learn about our development process, how to propose fixes and improvements, and how to build and test the code.

\subsection*{Special Thanks}

{\itshape To be completed...}

\subsection*{License}

Apache License 2.\+0 